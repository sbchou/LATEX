% this environment should probably be called abstract,
% but we want people to also be able to get at the more
% basic abstract environment
\def\abstractpage{\cleardoublepage
\begin{center}{\large{\bf \@title} \\
by \\
\@author \\[\baselineskip]}
\par
\def\baselinestretch{1}\@normalsize
Submitted to the \@department, \\
\@school \\
on \@thesisdate, in partial fulfillment of the \\
requirements for the \@degreeword\ of \\
\@degree
\end{center}
\par
\begin{abstract}}
This thesis uses mixed methods and datasets to explore how political news is perceived and shared within and across party lines in the context of the 2016 US presidential elections. We begin by examining the impact of political context versus article content on the reader through a crowdsourced study, and follow up with a large scale analysis of story sharing on the social platform Twitter to find cases where popularity transcends political affiliation.
 
In Part One, we look at reader \emph{reactions}. We investigate the question of trust in political news by performing a study online. We look at the impact of content features (reading level of the article) versus context clues (media brands) to find that political affiliation and brand outweigh all other aspects.

%We find that reading level has no significant impact on whether or not political news is seen as trustworthy, and that media brand, as well as party loyalty, matters above all other aspects in biasing the reader. This assertion holds when \emph{the content itself remains constant}, and the same news story is shown as attributed to different media outlets, resulting in different levels of trust. 

In the second part of this thesis, we focus on reader \emph{actions}.  In particular, we look at how political news stories from the same time period are shared on the social media platform Twitter. As we found party loyalty and media brand perceptions to be significant influences on the reader's opinion of news, we are especially interested in examining emotional features that cause stories to become popular beyond political boundaries. 

%Extending previous studies relating different emotional responses and virality, we look for text that might trigger an emotional response in the reader. 

%We find that the degree of emotionality in a story as well as the direction of sentiment transcends affiliation and degree of political engagement in sharing behavior. 

Together, these two parts hope to form a more complete view of factors affecting and driving readers in an election cycle that is heavily influenced by media coverage, both traditional and new.





\afterpage{\blankpage}

 