% this environment should probably be called abstract,
% but we want people to also be able to get at the more
% basic abstract environment
\def\abstractpage{\cleardoublepage
\begin{center}{\large{\bf \@title} \\
by \\
\@author \\[\baselineskip]}
\par
\def\baselinestretch{1}\@normalsize
Submitted to the \@department, \\
\@school \\
on \@thesisdate, in partial fulfillment of the \\
requirements for the \@degreeword\ of \\
\@degree
\end{center}
\par
\begin{abstract}}

 
This thesis explores readers' perceptions of political news coverage in the 2016 US presidential elections. We approach the question of the impact of content versus context in biasing the reader from the lens of data both big and small.


In the first part, we explore the question of \emph{trust} in political news by performing a study that analyzes the impact of content features (reading level of the article) versus context clues (media brands). We find that reading level has no significant impact on whether or not readers find political news fair, and that media brand, as well as candidate loyalty, matters above all other aspects in influencing the reader's perceptions of bias. This assertion holds when \emph{the content itself remains constant}, and the same news story is shown as attributed to different media outlets, resulting in different levels of trust. 

In the second part of this thesis, we extrapolate our findings to examine how political news stories from the same time period are shared on the social media platform Twitter. Again, we explore the question of content (topic, sentiment, emotionality of the writing) versus context (media brand, political party) in influencing a reader. Sharing news stories requires a level of activation on the part of the reader beyond passive readership. As we found candidate loyalty and media brand perceptions to be significant influences on the reader's opinion of news, we are particularly interested in examining the content and context features of stories that become viral beyond party boundaries.

Together, these two parts hope to form a more complete view of factors affecting and driving readers in an election cycle that is heavily influenced by media coverage, both traditional and new.


\afterpage{\blankpage}

 