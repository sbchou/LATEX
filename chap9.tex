
\chapter{Analysis}
In this chapter, we test for correlations between the metrics defined in Chapter 9 and popularity of a story on Twitter.

\section{Hypotheses}
Re-iterating from Chapter 7, the focus of Part II of this thesis is to examine how emotionally evocative text in an article relates to the way it is shared on Twitter. In addition, we look at the effects of story length on its popularity as a baseline to confirm theories of attention on the Internet.

 
We hypothesize the following behavior in our dataset of stories and tweets:

\begin{itemize} 
    \item \textbf{H1:} Story length has a \emph{negative} correlation with Twitter shares, due to the effects of the Internet attention economy and overexposure to political media \cite{goldhaber1997attention}.
    \item \textbf{H2:} Emotionality has a \emph{positive} correlation with Twitter shares, consistent for viral content in general \cite{berger2012makes}.
    \item \textbf{H3:} Positivity has a \emph{negative} correlation with Twitter shares, due to the nature of political news and contrary to generalized findings \cite{berger2012makes}

\end{itemize}

For each of these three independent variables (story length, emotionality, positivity) we repeat analyses across three views of the data: first, the entire dataset; then, by political candidate followed amongst users who follow only one candidate; and finally, by the number of political candidates followed (degree of political engagement), to look for differences amongst different populations of political tweeters.


\section{Methodology}

Since our dependent variable, tweet volume, is a set of discrete counts that are positively truncated, we use negative binomial regression models for our analysis \cite{scott1997regression}. The distribution of tweet volume is not a normal distribution, and it is not recommended to perform a log transformation on count data to fit it to an OLS regression unless there is little dispersion in the data \cite{o2010not}. Poisson models are a subset of negative binomial models without the dispersion parameter. Below, we see that the negative binomial model provides the best fit and that our data is overdispersed, as the dispersion parameter $\theta$ is greater than 1.   

In each case, we compare our findings to those using plain linear and Poisson regression models and are able to achieve the same significant results.  

\section{All Data}
 
% Table created by stargazer v.5.2 by Marek Hlavac, Harvard University. E-mail: hlavac at fas.harvard.edu
% Date and time: Sun, Jul 24, 2016 - 17:29:16
\begin{table}[!htbp] \centering 
  \caption{Tweet Volume vs. Story Length, All Data} 
  \label{} 
    \begin{tabular}{@{\extracolsep{5pt}}lccc} 
    \\[-1.8ex]\hline 
    \hline \\[-1.8ex] 
     & \multicolumn{3}{c}{\textit{Dependent variable:}} \\ 
    \cline{2-4} 
    \\[-1.8ex] & \multicolumn{3}{c}{Number of Tweets} \\ 
    \\[-1.8ex] & \textit{OLS} & \textit{Poisson} & \textit{negative} \\ 
     & \textit{} & \textit{} & \textit{binomial} \\ 
    \\[-1.8ex] & (1) & (2) & (3)\\ 
    \hline \\[-1.8ex] 
     Story length & $-$0.004$^{***}$ & $-$0.0001$^{***}$ & $-$0.0001$^{***}$ \\ 
      & (0.001) & (0.00000) & (0.00002) \\ 
      & & & \\ 
     Constant & 50.426$^{***}$ & 3.936$^{***}$ & 3.895$^{***}$ \\ 
      & (1.819) & (0.005) & (0.027) \\ 
      & & & \\ 
    \hline \\[-1.8ex] 
    Observations & 2,650 & 2,650 & 2,650 \\ 
    R$^{2}$ & 0.003 &  &  \\ 
    Adjusted R$^{2}$ & 0.003 &  &  \\ 
    Log Likelihood &  & $-$65,042.390 & $-$12,778.740 \\ 
    $\theta$ &  &  & 1.346$^{***}$  (0.034) \\ 
    Akaike Inf. Crit. &  & 130,088.800 & 25,561.490 \\ 
    Residual Std. Error & 60.156 (df = 2648) &  &  \\ 
    F Statistic & 8.452$^{***}$ (df = 1; 2648) &  &  \\ 
    \hline 
    \hline \\[-1.8ex] 
    \textit{Note:}  & \multicolumn{3}{r}{$^{*}$p$<$0.1; $^{**}$p$<$0.05; $^{***}$p$<$0.01} \\ 
    \end{tabular} 
\end{table} 
\newpage
 

% Table created by stargazer v.5.2 by Marek Hlavac, Harvard University. E-mail: hlavac at fas.harvard.edu
% Date and time: Sun, Jul 24, 2016 - 18:56:10
\begin{table}[!htbp] \centering 
  \caption{Tweet Volume vs. Emotionality, All Data} 
  \label{} 
    \begin{tabular}{@{\extracolsep{5pt}}lccc} 
    \\[-1.8ex]\hline 
    \hline \\[-1.8ex] 
     & \multicolumn{3}{c}{\textit{Dependent variable:}} \\ 
    \cline{2-4} 
    \\[-1.8ex] & \multicolumn{3}{c}{Number of Tweets} \\ 
    \\[-1.8ex] & \textit{OLS} & \textit{Poisson} & \textit{negative} \\ 
     & \textit{} & \textit{} & \textit{binomial} \\ 
    \\[-1.8ex] & (1) & (2) & (3)\\ 
    \hline \\[-1.8ex] 
     Emotionality & 305.229$^{**}$ & 6.111$^{***}$ & 6.019$^{***}$ \\ 
      & (122.427) & (0.276) & (1.777) \\ 
      & & & \\ 
     Constant & 40.349$^{***}$ & 3.714$^{***}$ & 3.716$^{***}$ \\ 
      & (2.685) & (0.006) & (0.039) \\ 
      & & & \\ 
    \hline \\[-1.8ex] 
    Observations & 2,650 & 2,650 & 2,650 \\ 
    R$^{2}$ & 0.002 &  &  \\ 
    Adjusted R$^{2}$ & 0.002 &  &  \\ 
    Log Likelihood &  & $-$65,180.470 & $-$12,779.510 \\ 
    $\theta$ &  &  & 1.345$^{***}$  (0.034) \\ 
    Akaike Inf. Crit. &  & 130,364.900 & 25,563.030 \\ 
    Residual Std. Error & 60.181 (df = 2648) &  &  \\ 
    F Statistic & 6.216$^{**}$ (df = 1; 2648) &  &  \\ 
    \hline 
    \hline \\[-1.8ex] 
    \textit{Note:}  & \multicolumn{3}{r}{$^{*}$p$<$0.1; $^{**}$p$<$0.05; $^{***}$p$<$0.01} \\ 
    \end{tabular} 
\end{table} 
\newpage
 
% Table created by stargazer v.5.2 by Marek Hlavac, Harvard University. E-mail: hlavac at fas.harvard.edu
% Date and time: Sun, Jul 24, 2016 - 18:58:37
\begin{table}[!htbp] \centering 
  \caption{Tweet Volume vs. Positivity, All Data} 
  \label{} 
    \begin{tabular}{@{\extracolsep{5pt}}lccc} 
    \\[-1.8ex]\hline 
    \hline \\[-1.8ex] 
     & \multicolumn{3}{c}{\textit{Dependent variable:}} \\ 
    \cline{2-4} 
    \\[-1.8ex] & \multicolumn{3}{c}{Number of Tweets} \\ 
    \\[-1.8ex] & \textit{OLS} & \textit{Poisson} & \textit{negative} \\ 
     & \textit{} & \textit{} & \textit{binomial} \\ 
    \\[-1.8ex] & (1) & (2) & (3)\\ 
    \hline \\[-1.8ex] 
     Positivity & $-$281.010$^{**}$ & $-$6.029$^{***}$ & $-$5.391$^{***}$ \\ 
      & (139.546) & (0.338) & (2.029) \\ 
      & & & \\ 
     Constant & 47.078$^{***}$ & 3.851$^{***}$ & 3.849$^{***}$ \\ 
      & (1.221) & (0.003) & (0.018) \\ 
      & & & \\ 
    \hline \\[-1.8ex] 
    Observations & 2,650 & 2,650 & 2,650 \\ 
    R$^{2}$ & 0.002 &  &  \\ 
    Adjusted R$^{2}$ & 0.001 &  &  \\ 
    Log Likelihood &  & $-$65,253.640 & $-$12,781.640 \\ 
    $\theta$ &  &  & 1.343$^{***}$  (0.034) \\ 
    Akaike Inf. Crit. &  & 130,511.300 & 25,567.270 \\ 
    Residual Std. Error & 60.206 (df = 2648) &  &  \\ 
    F Statistic & 4.055$^{**}$ (df = 1; 2648) &  &  \\ 
    \hline 
    \hline \\[-1.8ex] 
    \textit{Note:}  & \multicolumn{3}{r}{$^{*}$p$<$0.1; $^{**}$p$<$0.05; $^{***}$p$<$0.01} \\ 
    \end{tabular} 
\end{table} 
\newpage



\section{By Degree of Political Engagement}

In the our analyses, we segment levels of political engagement into three groups for the sake of comparison:

\begin{itemize}
  \item \emph{The unaffiliated} (those who follow no presidential candidates, but do tweet about political news)
  \item \emph{Single-candidate} followers (those who follow one and only one presidential candidate, and tweet about political news)
  \item \emph{Political aficionados} (those who follow all 4 (or more) candidates, and tweet about political news)
\end{itemize}

We repeat the same methods and variables in determining our correlations. For the sake of brevity, all tables and comparisons of all three models (OLS, Poisson, NB model) can be found in Appendix A, tables A.1 - A.9.

Overall, we find that: 

\emph{Unaffiliated tweeters} show the same patterns as the general dataset with a negative correlation between story length and Twitter shares ($\beta=-0.003$, $p<0.01$), positive correlation between emotionality and Twitter shares ($\beta=7.427$, $p<0.01$), and a negative correlation between positivity and Twitter shares ($\beta=-4.036$, $p<0.01$).

\emph{Single-candidate followers}, on the other hand, show a slight positive correlation between story length and number of Twitter shares ($\beta=0.0002$, $p<0.01$). We hypothesize that if following a single candidate can serve as a proxy for candidate loyalty, then perhaps the correlation signifies a willingness to read and share more complex content on behalf of the candidate and a deeper degree of political involvement.

We see the same effects for the \emph{political aficionados} group, again, a small but significant positive correlation between story length and number of tweets ($\beta=0.0003$, $p<0.01$). Again, this suggests a potential difference in levels of engagement with political news.


\section{By Candidate} 

\subsection{Trump-Only Followers}


\subsection{Clinton-Only Followers}
 

\subsection{Sanders-Only Followers}


\subsection{Cruz-Only Followers}



% \section{Extra: Any relation with Trust?}
% % see jun 10 update
% % because it'd be funny not to check











