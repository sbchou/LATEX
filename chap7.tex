\chapter{Taking Media Analysis to Another Dimension of Data}
In the second part of this thesis, we extend our findings from Part One to examine how political news stories from the same time period are shared on social media. In the current digital age, broadcasting a link to friends and peers often emerges as the next logical step to reading and reacting to the news. We will focus our analysis to the social platform Twitter due to its popularity for disseminating news stories. A Pew Research survey from August 2015 showed that nearly two-thirds of adults in the U.S. who are on Twitter use the platform to get news.  Government and politics emerges as the third most popular type of news to tweet about (following entertainment and sports), with 17\% of tweets from the average user being about that subject \cite{pew-twitter-news}.

Sharing news stories requires a level of activation on the part of the reader beyond passive readership; often, this trigger is emotional in nature. In 2011, researchers from the Wharton School found that the potential for a news story to go viral is partially driven by physiological arousal. By performing a large-scale analysis of articles from the New York Times coded for emotional content, they were able to find emotional factors in the text that predicted its potential to make the newspaper’s “most-emailed” list \cite{berger2012makes}.

The following chapters of this thesis examine this emotional relationship for political news stories covering the 2016 U.S. elections. From our studies in Part One, we were able to see that political interactions between the reader and the source of the story proved to be by far the strongest motivator of trust (or lack thereof!) in reading the news. Now, in Part Two, we look for the emotional triggers in content that transcend political boundaries in motivating readers to spread the news.


\section{The Social Media Megaphone}
In the changing landscape of both journalism and politics, social media is playing an increasingly large role in mobilizing and spreading information to citizens. Many attribute President Barack Obama’s win in 2008 as the first example of a successful social media campaign in the elections. Establishing an online presence that recruited more than 3 million individual contributors and 5 million volunteers, Obama created a grassroots political movement \cite{cogburn2011networked}. Publicity and public sound bites matter-- especially when it’s free and has the potential to go viral.

This election cycle, in particular, already shows a heavy skew by social media. The New York Times estimated a 2 billion-dollar advantage in free media for Donald Trump on platforms from television to Twitter, all of which has no small impact on the messages broadcast to voters \cite{nyt-trump-free-media}. Although ``free media'' messages have less ability to be carefully controlled in comparison to paid advertisements, they also have more potential to reach a wider audience. Sentiments echoed by one potential voter now has the ability to be broadcast and spread to millions of others in a real-time, public sphere.
  


\section{Trust, Virality, and Controversy}

%\section{Conclusions}
So,
Why does this matter?
Well, in addition to sharing new content, social media acts as a megaphone for other (traditional) media. Finding these patterns between how people observe the trustworthiness of stories in a experimental setting and how they are shared on twitter can give insights to social media studies of the election, as we are doing in our group.


















