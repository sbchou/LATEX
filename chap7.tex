\chapter{Taking Media Analysis to Another Dimension of Data}
In the second part of this thesis, we extend our findings from Part One to examine how political news stories from the same time period are shared on social media. In the current digital age, broadcasting a link to friends and peers often emerges as the next logical step to reading and reacting to the news. We will focus our analysis to the social platform Twitter due to its popularity for disseminating news stories. A Pew Research survey from August 2015 showed that nearly two-thirds of adults in the U.S. who are on Twitter use the platform to get news.  Government and politics emerges as the third most popular type of news to tweet about (following entertainment and sports), with 17\% of tweets from the average user being about that subject \cite{pew-twitter-news}.

Sharing news stories requires a level of activation on the part of the reader beyond passive readership; often, this trigger is emotional in nature. In 2011, researchers from the Wharton School found that the potential for a news story to go viral is partially driven by physiological arousal. By performing a large-scale analysis of articles from the New York Times coded for emotional content, they were able to find emotional factors in the text that predicted its potential to make the newspaper’s “most-emailed” list \cite{berger2012makes}.

The following chapters of this thesis examine this emotional relationship for political news stories covering the 2016 U.S. elections. From our studies in Part One, we were able to see that political interactions between the reader and the source of the story proved to be by far the strongest motivator of trust (or lack thereof!) in reading the news. Now, in Part Two, we look for the emotional triggers in content that transcend political boundaries in motivating readers to spread the news.




\section{The Social Media Megaphone}
cite lit that says most people use twitter for sharing articles.
group. 

also cite (prev?) piece about the huge impact of social media in this election. But also barack's election too.



As of early 2015, 63\% of Facebook and Twitter users get news on their respective sites. This is up substantially from 2013, when about half of each social network’s users (47\% for Facebook and 52\% for Twitter) reported getting news there.

Use of Twitter for news, for example, grew among both users under 35 (55\% to 67\%) and those ages 35 and older (47\% to 59\%). \cite{Pew-news-sharing}


\section{Trust, Virality, and Controversy}

%\section{Conclusions}
So,
Why does this matter?
Well, in addition to sharing new content, social media acts as a megaphone for other (traditional) media. Finding these patterns between how people observe the trustworthiness of stories in a experimental setting and how they are shared on twitter can give insights to social media studies of the election, as we are doing in our group.


















