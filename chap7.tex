\chapter{Taking Media Analysis to Another Dimension of Data}
In the second part of this thesis, we extend our findings from Part One to examine how political news stories from the same time period are shared on social media. In the current digital age, broadcasting a link to friends and peers often emerges as the next logical step to reading and reacting to the news. We will focus our analysis to the social platform Twitter due to its popularity for disseminating news stories. A Pew Research survey from August 2015 showed that nearly two-thirds of adults in the U.S. who are on Twitter use the platform to get news.  Government and politics emerges as the third most popular type of news to tweet about (following entertainment and sports), with 17\% of tweets from the average user being about that subject \cite{pew-twitter-news}.

Sharing news stories requires a level of activation on the part of the reader beyond passive readership; often, this trigger is emotional in nature. In 2011, researchers from the Wharton School found that the potential for a news story to go viral is partially driven by physiological arousal. By performing a large-scale analysis of articles from the New York Times coded for emotional content, they were able to find emotional factors in the text that predicted its potential to make the newspaper’s “most-emailed” list \cite{berger2012makes}.

The following chapters of this thesis examine this emotional relationship for political news stories covering the 2016 U.S. elections. From our studies in Part One, we were able to see that political interactions between the reader and the source of the story proved to be by far the strongest motivator of trust (or lack thereof!) in reading the news. Now, in Part Two, we look for the emotional triggers in content that transcend political boundaries in motivating readers to spread the news.


\section{The Social Media Megaphone}
In the changing landscape of both journalism and politics, social media is playing an increasingly large role in mobilizing and spreading information to citizens. President Barack Obama’s win in 2008 is often attributed as the first example of a successful social media campaign in the elections. Establishing an online presence that recruited more than 3 million individual contributors and 5 million volunteers, Obama created a grassroots political movement \cite{cogburn2011networked}. Publicity and public sound bites matter-- especially when it’s free and has the potential to go viral.

This election cycle, in particular, already shows a heavy skew by social media. The New York Times estimated a 2 billion-dollar advantage in free media for Donald Trump on platforms from television to Twitter, all of which has no small impact on the messages broadcast to voters \cite{nyt-trump-free-media}. Although ``free media'' messages have less ability to be carefully controlled in comparison to paid advertisements, they also have more potential to reach a wider audience. Sentiments echoed by one potential voter now has the ability to be broadcast and spread to millions of others in a real-time, public sphere.
  
\section{The (Short) Attention Economy}
At the same time that social media has the power to create a flood of free advertising and media for political candidates, the abundance of information on the web has created new challenges and questions about the kind of content being processed by readers. This paradox-- between the ease of accessibility to information and the increasingly limited bandwidth of consumers-- is described as one of the challenges of being in an \emph{attention economy} \cite{goldhaber1997attention}. Moreover, high-impact events like the presidential elections especially intensifies this effect-- about 60 \% of Americans reported feeling exhausted by media coverage of the elections in July of 2016 \cite{election-fatigue}. To explore the effects of the attention economy on the reading of political news, we examine \textbf{story length} and how it relates to sharing popularity in the analysis to follow.


\section{Negativity in Politics and the Internet}
In addition, the option of anonymity and pseudo-anonymity on a social network like Twitter (along with other traits of Internet communication), is theorized to contribute to increased negative and hostile behavior, potentially increasing tension for the already-fraught subject of politics. This phenomenon, is coined as the \emph{online disinhibition effect} \cite{suler2004online}. 

In Berger and Milkman’s study of story virality, it was found that \emph{positive} content was more likely to be shared than negative content-- against conventional belief \cite{berger2012makes}. Political news, however, is a unique category of news, and this election in particular-- where one-in-four Americans report disliking the presidential candidates-- appears to have a negative overtone.

To compare the sharing of election news stories versus patterns of general virality in the news, and to examine the extent in which negative sentiment is popular, we calculate the \emph{negativity} of stories, and how that relates to Twitter behavior.

In addition, we examine the effects of the degree of combined emotionality in the content and how that relates to Twitter shares, to see if either more positive or more negative content is more likely to be shared overall than content that ranks low in emotionality. Although positive content was found to be more popular than negative content in the sharing of stories, both highly positive and highly negative content was more likely to become viral, and we expect the same to hold for political news \cite{berger2012makes}. 

The calculation of both metrics are detailed below in Section 8.2.












