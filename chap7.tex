\chapter{Main Study}

%\section{Motivation}

From our exploratory study, we were able to obtain a significant but weak effect between disclosing the source and the levels of trust marked by readers towards an article.

We also observed trends that suggested an interaction between disclosing the source and the reading level of a story, however we did not obtain enough samples to show a statistically significant result. 

Furthermore, multiple levels of independent variables (ie: 5 levels for input source) made modeling complex and the results less clear.

The dataset was also unbalanced and sparse (ie, because of large numbers of input variables we did not have complete representation for each category, such as high, low, and mid-reading level stories for every outlet and topic). We tried to control for those factors by randomization, however it made more difficult to analyze specific correlations between source and trust.

To further explore the interaction between disclosing the source and the reading level of the story, we set up another crowdsourcing experiment on CrowdFlower, this time targeting this specific interaction, to see if there is a significant effect between the two.


\section{Experimental Design}

We revised the experiment to have a 4 x 2 subject design.
Within subject: Candidates and High/Low Reading Level
Mixed design


\begin{center}
\begin{tabular}{ | m{10em} | m{7em}| m{7em} | m{7em} | m{7em} | } 
 \hline
  & \textbf{Source: None} & \textbf{Source: AP} & \textbf{Source: Fox} & \textbf{Source: CNN} \\
 \hline
 \textbf{High Reading Level} & Clinton, Cruz, Sanders, Trump & Clinton, Cruz, Sanders, Trump & Clinton, Cruz, Sanders, Trump & Clinton, Cruz, Sanders, Trump  \\ 
 \textbf{Low Reading Level} & Clinton, Cruz, Sanders, Trump & Clinton, Cruz, Sanders, Trump & Clinton, Cruz, Sanders, Trump & Clinton, Cruz, Sanders, Trump \\ 
 \hline
\end{tabular}
\end{center}

%What stayed the same? What changed?

\section{Data Selection} 
%[How did you choose the eight stories?]


%Why did you choose trust and fairness?

 
% How did you control for quality?

  
% What kind of people signed up for your study?

% How did you recruit them? What was their incentive?

% What kind of effects were you looking for?
 
% What kind of effects did you find?

% How do your trustworthiness findings line up with the findings from Pew surveys and prior work?

% What were potential limitations of the study?

% Analysis

\section{Limitations}

% % This is an example of how you would use tgrind to include an example
% % of source code; it is commented out in this template since the code
% % example file does not exist.  To use it, you need to remove the '%' on the
% % beginning of the line, and insert your own information in the call.
% %
% %\tagrind[htbp]{code/pmn.s.tex}{Post Multiply Normalization}{opt:pmn}
%  