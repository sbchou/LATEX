\chapter{Conclusion}

\section{Findings}

\section{Potential Impact}

\section{Limitations}
We found these effects, so what?
 
\subsection{Data Collection}
* obviously sources are a small set, might not be representative
* matching tweets and stories this way using regex is fast but might not be complete

 

%%% this is part I conlcusion ! move up!
% Media distrust, which has intensified over the last two decades, is a phenomenon with serious implications in the practice of democracy and a well-informed public.  

% In an election year prefaced by deep cynicism towards American institutions (a 2015 survey showed that just 19\% of the population trusts the federal government), attitudes towards the news media fare no better. Almost two-thirds of Americans think that the national news media is a negative influence on the country \cite{beyond-distrust}. 

% Our results confirm previous hypotheses about the importance of the role of the reader in determining news trust and bias \emph{over} the role of the content itself. We show that drastically different use of language complexity has little effect on the reader in comparison to the presence of media brands.

% Interestingly, although showing media from outlets of opposing political orientations decreases trust and fairness perceptions in the reader, when no outlet is attributed to the story, trust in the story also decreases. This suggests that interventions designed to \emph{level} news stories by aggregating them without attribution to source might not be effective in creating a balanced news diet for sustaining informed voters. 