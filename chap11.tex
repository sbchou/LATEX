\chapter{Final Remarks}
If a free press is central to a functioning democracy, the term ``free'' has taken on a whole new dimension in the 21st century. When Jürgen Habermas conceptualized the idea of the ``public sphere'' in 1962 as a middle ground between private individuals and governing authorities where ideas could be exchanged and discussed, it was hardly imaginable that this space would one day be virtual \cite{habermas1991structural}.

Yet the Internet and social media have quickly become battle grounds in lieu of the coffeeshops Habermas cited for a new era of political discussion and debate, satire and mockery. At the same time, influence of traditional media and journalism continues to have a significant impact on how voters gather information about politics. Just two months into the election year, more than 90\% of Americans stated that they had learned about the election from at least one publication in the previous week when surveyed \cite{election-fatigue}. 

In light of the rising importance of social media as well as the continued reach of traditional media coverage, this thesis analyzed readers' perceptions of political news in both experimental and ``organic'' observations. We take both  a ``small data'' and ``big data'' approach to first study factors influencing percieved media bias and news trust, and then examine what kind of political news becomes popular on Twitter. 

Together, we hope to plant seeds for future directions in traditional and social media studies, and shed light on readers' reactions to a unique and emotionally charged election year.