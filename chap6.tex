\chapter{Taking Trust to Another Dimension of Data}

We found this.
So what?

One can't study traditional media (focus of this study) without acknowledging the huge role social media had and is having on the 2016 US Elections. (Again, cite NYT piece on free media, Trump and Sander's social media power).

In this section, we extrapolate from our studies to another social sphere: the public sphere of Twitter. 



\section{The Social Media Megaphone}
cite lit that says most people use twitter for sharing articles.
group. 



As of early 2015, 63\% of Facebook and Twitter users get news on their respective sites. This is up substantially from 2013, when about half of each social network’s users (47\% for Facebook and 52\% for Twitter) reported getting news there.

Use of Twitter for news, for example, grew among both users under 35 (55\% to 67\%) and those ages 35 and older (47\% to 59\%). \cite{Pew-news-sharing}


\section{Trust, Virality, and Controversy}

\section{Conclusions}

Why does this matter?
Well, in addition to sharing new content, social media acts as a megaphone for other (traditional) media. Finding these patterns between how people observe the trustworthiness of stories in a experimental setting and how they are shared on twitter can give insights to social media studies of the election, as we are doing in our group.


















