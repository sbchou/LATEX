\chapter{The Power of (Percieved) Media Bias}




\section{The Effects of Media Bias}

%Why is media bias IMPORTANT? Why is the problem IMPORTANT?

 
%- Fox News Effect
%- Does the media matter


\section{The Role of the Reader in Perceptions of Bias}

It comes as no surprise that our own political stances have a significant effect in our perceptions of bias in the media. 

In even seemingly neutral stories, partisans tend to view reporting as biased against their own views. This phenomenon-- deemed the ``hostile media effect''-- was first studied at Stanford University by Robert P. Vallone, Lee Ross, and Mark R. Lepper in 1985 \cite{vallone1985hostile}. Although ``true'' neutrality of a story is nearly impossible to quantify due to the subjective nature of the concept, Vallone et. al were able to successfully demonstrate that partisans of \emph{both} sides (pro-Israeli and pro-Arab) viewed the same news segments as hostile towards their beliefs and favorable to the other side.

%? Perceptions of media bias, then, have as much to do as self-serving motivations to secure preferential treatment as they do with the media itself.
  
%? The political leanings of the reader are essential considerations when attempting to measure other factors that contribute to bias. In 

\section{The Role of Media Brands in Perceptions of Bias}
%The media, of course, is not just one unified mass, and in an increasingly fragmented ecosystem, the role of media brands is a crucial factor in the perception of bias. Although most research
%\cite{baum2008eye}


 
%For instance, most research on the hostile media phenomenon conceptualizes the news media as an undifferentiated mass of information sources that individuals can (and do) reasonably characterize as having a uniform political orientation (Giner-Sorolla and Chaiken 1994, Peffley et al. 2001, Eveland and Shah 2003). Yet, the past two decades have seen a dramatic increase in the number and variety of news sources. One consequence is that Democrats and Republicans are increasingly likely to differ systematically in their assessments of specific media outlets.






%With the decline of print newspapers, a diverse number of new platforms and web-centric publications have risen.


%How do you control for the above things in your study?

\section{The Role of Language [Policial Persuasion]}




\subsection{Language and Politics}
%Presidential speeches degrading over time-- ie simple language appeals to the masses in politics
\subsection{The Seductive Allure [... of Simple] Language}
%But we trust complex language for explaining technical facts

%Test image

%\includegraphics[width=\textwidth]{flowchart_final}


\section{Importance in Political Outcomes}
Fox news effect 

\section{The 2016 Elections} 
\subsection{Criticism of Media Bias} 
%(Obama Speech)

%So.... are you what you cover?








