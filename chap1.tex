%% This is an example first chapter.  You should put chapter/appendix that you
%% write into a separate file, and add a line \include{yourfilename} to
%% main.tex, where `yourfilename.tex' is the name of the chapter/appendix file.
%% You can process specific files by typing their names in at the 
%% \files=
%% prompt when you run the file main.tex through LaTeX.

\chapter{Introduction} 

Does anyone trust the news anymore? Not according to the latest Gallup Poll, which showed that only 4 in 10 Americans believe that mass media does a good job of reporting the news ``fully, fairly, and accurately.'' It's a major decline since the poll was first taken in 1999, back when more than half (55\%) of Americans believed the news was trustworthy \cite{Gallup-trust-2015}.

And the trend has been steadily downward: in short, the majority of Americans have had little to no trust in mass media news coverage since 2007: a discouraging view for a tumultuous time in journalism.

But beyond frustrated readers and reporters, why does distrust in the news matter? For one, media bias---or at the very least, the \emph{belief of} a biased media bias---may have a significant impact on the practice of democracy. A 2006 study from Georgetown University shows that those with more negative attitudes towards the news tend to be more highly influenced by their partisan prior beliefs and less by contemporary issues and messages when voting \cite{ladd2005attitudes}. This implies that distrust of media plays a large role in the polarization of American politics.

In light of the upcoming 2016 elections, this thesis explores perceptions of media trust in coverage of the presidential candidates. Claims of media bias and favoritism are especially high-stakes in election years, where trust has been shown to plummet \cite{Gallup-trust-2015}. In this election cycle, cries of bias have been especially loud: Analysis at the New York Times showed that the news media gave Republican candidate Donald Trump a \$1.9 billion advantage in free publicity, an amount 190 times as much as paid advertising \cite{Trump-advantage}.

In this thesis, we examine some of the factors that contribute to the perception of media bias. In particular, how does the \emph{content} of a story (reading level and vocabulary) affect the reader versus the \emph{context} (publication and author)? 

We funnel the larger question of media bias into one primary dimension: media trust, and examine the role of simple and complex language in influencing the reader's decision to trust or distrust reporting. In addition, we measure perceptions of fairness and favorability. 

%We break down the larger question of media bias in two dimensions: trust and fairness in reporting, and examine the role of language in influencing the reader.  
Although studies have been conducted to both examine the psychological effect of wording on believability and the impact of media brands and bias, separating and comparing these two factors remains largely unexamined \cite{weisberg2008seductive, baum2008eye}

%Two studies are performed: the first an exploratory one focusing on detecting reading level effects, and the second a follow-up on media brand effects, to collect reader's perceptions of news stories through crowdsourcing. 
To test our hypotheses on news trust, we perform a study on the crowdsourcing platform CrowdFlower. We manipulate the source of the story to examine effects of media brands on the reader, and also compare trust and fairness rankings between high and low reading level stories.

From our findings in an experimental setting, we then extrapolate to another dimension of media: the social sphere of Twitter. We look for trends between news stories that result in polar perceptions of trustworthiness and the patterns in which people share them, to help form a comprehensive view of the effects of media trust and distrust on behavior.

Although the general consensus of mistrust is clear, perception of media bias is a complex phenomenon to dissect, as it combines social and psychological effects with the traits of the story itself. This thesis hopes to shed new light on understanding what motivates readers' trust and distrust of news media, and pave pathways for positive intervention as well as future studies on story reading and sharing.
 

 