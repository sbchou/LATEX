%% This is an example first chapter.  You should put chapter/appendix that you
%% write into a separate file, and add a line \include{yourfilename} to
%% main.tex, where `yourfilename.tex' is the name of the chapter/appendix file.
%% You can process specific files by typing their names in at the 
%% \files=
%% prompt when you run the file main.tex through LaTeX.
\chapter{Introduction}
Most Americans say that they want to read news that's unbiased. A survey from Pew Research in 2012 showed that more than two-thirds (68\%) of readers want to read political articles with a neutral stance, compared to just a little less than a quarter (23\%) of those who want to read those stories that share their point of view.1 But what exactly does that mean?

To begin with, whether or not we perceive news as biased is biased in itself. Conservative readers tend to view media as more biased than both Democrats and Independents (49\% to 32\% and 35\%, respectively)\cite{Pew-bias-2012}.  

The Hostile Media Effect, first studied by Vallone, Ross, and Lepper in 1985, gives one possible explanation for discrepancies: it describes a phenomenon where people with strong stances on an issue tend to perceive media covered as biased against their opinions, even on the same article.2

Clearly, finding bias in news depends on who the reader is as much as what they are reading. 

In my thesis, I seek to examine the effects of context versus content in perceptions of media bias. In particular, when the context of a story is removed, how do linguistic features, in particular reading level and vocabulary, in the content affect the reader? Although studies have been conducted to both examine the psychological effect of wording on believability (see ``Seductive Allure'') and the impact of media brands and bias (see Baum, 2008), I seek to combine and contrast the two.

To do so, I will perform an A/B study for a broad range of readers to read and annotate political news stories (collected daily and sorted using a machine learning classifier). Each story is determined to be primarily about one political candidate and one topic computationally. In the control group, readers are given the full text of the article with no additional content. In the experimental group, readers are given a link to the original article complete with the byline, publication, and images. Stories are classified as either ``high reading level,'' ``average reading level,'' or ``low reading level'' by the Flesch-Kincaid test.

For each reader, I will collect their  demographic information, and self-reported political stances. I will then analyze the effects of reading level versus media brand in the reader's perception of the article. 

I want to measure just how strong the effect of the media brand and the reader's beliefs are.
