%% This is an example first chapter.  You should put chapter/appendix that you
%% write into a separate file, and add a line \include{yourfilename} to
%% main.tex, where `yourfilename.tex' is the name of the chapter/appendix file.
%% You can process specific files by typing their names in at the 
%% \files=
%% prompt when you run the file main.tex through LaTeX.
\chapter{Introduction}
 Most Americans say that they want to read news that's unbiased. A survey from Pew Research in 2012 showed that more than two-thirds (68\%) of readers want to read political articles with a neutral stance, compared to just a little less than a quarter (23\%) of those who want to read those stories that share their point of view \cite{Pew-bias-2012}. But what exactly does that entail?

To begin with, whether or not we see news as biased is biased in itself. On whole, conservative readers tend to view media as more biased than both Democrats and Independents (49\% to 32\% and 35\%, respectively)\cite{Pew-bias-2012}. Partisans have also been shown in experiment to view the news as antagonistic to their beliefs, a phenomenon known as the ``hostile media effect'' \cite{vallone1985hostile}. To further complicate issues, attempting to measure a story's ``true'' subjectivity is a slippery task. Can it ever be possible to measure subjectivity in an objective matter?
 
Yet the perception of media bias---regardless of ground truth---has a significant impact on the public. It can affect attitudes towards the government and even change voting outcomes, as measured by the impact of introducing Fox News in the cable market (it lead to a significant effect on the share of Republican votes in targeted areas) \cite{dellavigna2006fox}.

In light of the upcoming 2016 elections, this thesis explores perceptions of media bias towards presidential candidates. In particular, how does the \emph{content} of a story (reading level and vocabulary) affect the reader versus the \emph{context} (publication and author)? Although studies have been conducted to both examine the psychological effect of wording on believability and the impact of media brands and bias, separating and comparing these two factors remains largely unexamined \cite{weisberg2008seductive, baum2008eye}

The first part of this thesis consists of a crowdsourcing experiment to collect reader
's perceptions of candidate bias. In this section, we use A/B testing to measure the effect of viewing the source of a story and its context. We follow up with an analysis of story reading level, language and its impact.
 