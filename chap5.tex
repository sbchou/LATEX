\chapter{Findings}


% Table created by stargazer v.5.2 by Marek Hlavac, Harvard University. E-mail: hlavac at fas.harvard.edu
% Date and time: Thu, Apr 28, 2016 - 14:27:36
\begin{table}[!htbp] \centering 
  \caption{Factors Affecting Trust in News Articles} 
  \label{} 
\begin{tabular}{@{\extracolsep{5pt}}lccccc} 
\\[-1.8ex]\hline 
\hline \\[-1.8ex] 
Independent Variables & \multicolumn{1}{c}{$\beta$}\\ 
\hline \\[-1.8ex] 
Reading Level & 1,278 \\ 
Source (Fox News) & -0.45* \\ 
Reader Party (Unaffiliated) & -0.54.\\ 
%trust & 1,278 & 0.556 & 0.961 & $-$2 & 2 \\ 
\hline \\[-1.8ex] 
\emph{Note: N = 1,280. .p < 0.1. *p < .05. **p < .01. ***p < .001.}
\end{tabular} 
\end{table} 
 

 
\section{Reading Level Effects}






Note: this part is still brewing. I am working with one of Iyad's students to verify my statistical analysis and also trying to meet with Iyad next week just to make sure my experiment is correctly interpreted.

As of now, H1 and H2 are both not proved (looks like reading level has no significant effect). That also defaults H3 + H4 to be true.


\section{Media Brand Effects}

Preliminary results:

Using factorial ANOVA, when we model fairness as a function of story reading level $\times$ source of story $\times$ party of the candidate, we see that the source of the story has a high effect on the level of fairness perceived (F value 6.598, Pr (>F) 0.000201). Reading level has no significant effect. The same holds true for trust scores (F values 10.978, Pr(>F) 4.06e-07).

One interesting observation is that showing \emph{no source at all} has a negative effect on both trust and fairness. I am still analyzing that effect to see what it is.
 
Confirming hostile media effects, we see a significant effect of the reader's political affiliation aligning with the sources. We also so a significant effect if your candidate is being written about. 

I'm working to try to see those two effects in comparison, which would have interesting implications for this election cycle. Would be neat to see that party loyalty is officially (significantly) broken.
% COMPARE THE TWO

\section{Qualitative}
In our surveys, we left a space for people to leave comments about the task. Although most people did not fill out the question, here are some analyses of their responses.

To-do

\section{Conclusions}

To-do
%How do your trustworthiness findings line up with the findings from Pew surveys and prior work? What hypotheses did you verify from prior work?

\section{Limitations}

Our study shows significant effects that open potential new areas of experimentation while confirming past theories of how media bias is formed. In the interest of focus, our study centered around four candidates and a narrowed dataset of eight stories, but in the future could be replicated on a larger set of more diverse stories and outlets. 

Furthermore, although the contributor market on CrowdFlower is not representative of any specific region or demographic, it is also not representative of the nation at large.