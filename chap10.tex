\chapter{Conclusion}

\section{Findings}
 Overall, we verify the hypotheses that we propose in Chapter 9, and find that, on the whole:
\begin{itemize}
    \item Shorter stories are more likely to be shared on Twitter
    \item Stories high in emotional words, both negative and positive, are more likely to be shared on Twitter
    \item Stories that are less positive are more likely to be shared on Twitter.
\end{itemize}

These results confirm our expectations of reader attention on the Internet, the emotional nature of content virality, and the negative connotation of political media. 

However, we were unable to find many significant differences between segments by both the number of and specific political candidates that users followed.

This suggests that either the characteristics of political news we examine (length, emotionality, positivity) are universal in their effects on motivating readers to share articles, or that our methods of dividing Twitter users do not reveal significant underlying differences between readers.


\section{Limitations \& Future Work}
Although we take a ``big data'' approach to our analyses, our dataset is by no means a complete mapping of \emph{all} political news sharing activity on Twitter. Our sources are limited to a small but diverse set of publications, which are by no means representative of all news outlets.

Furthermore, we match tweets with stories optimizing for efficiency and speed by using regular expressions, rather than completeness.

Potential paths of future research include:
\begin{itemize}
\item Expanding the set of publications tracked (currently underway under the Electome project)
\item Replicating the analysis on a real-time basis
\item Using machine learning methods to match tweets with stories in a more intelligent way
\item Analyzing more nuanced emotive words in the text, in addition to positive and negative words
\item Looking at additional signals in the Twitter data, such as user characteristics of the sharer and network aspects of stories being shared
\item Segmenting Twitter users in different ways aside from candidates followed
 \end{itemize}

Still, our analysis provides a first view of article sharing on Twitter in a unique and eventful election year with large responses on social media.


%%% this is part I conlcusion ! move up!
% Media distrust, which has intensified over the last two decades, is a phenomenon with serious implications in the practice of democracy and a well-informed public.  

% In an election year prefaced by deep cynicism towards American institutions (a 2015 survey showed that just 19\% of the population trusts the federal government), attitudes towards the news media fare no better. Almost two-thirds of Americans think that the national news media is a negative influence on the country \cite{beyond-distrust}. 

% Our results confirm previous hypotheses about the importance of the role of the reader in determining news trust and bias \emph{over} the role of the content itself. We show that drastically different use of language complexity has little effect on the reader in comparison to the presence of media brands.

% Interestingly, although showing media from outlets of opposing political orientations decreases trust and fairness perceptions in the reader, when no outlet is attributed to the story, trust in the story also decreases. This suggests that interventions designed to \emph{level} news stories by aggregating them without attribution to source might not be effective in creating a balanced news diet for sustaining informed voters. 