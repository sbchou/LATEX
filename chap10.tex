
\chapter{Analysis}
In this chapter, we test for correlations between the metrics defined in Chapter 8 and popularity of a story on Twitter.

\section{Hypotheses}
Re-iterating from Chapter 7, the focus of Part II of this thesis is to examine how emotionally evocative text in an article relates to the way it is shared on Twitter. In addition, we look at the effects of story length on its popularity as a baseline to confirm theories of attention on the Internet.

For each of these three independent variables (story length, emotionality, positivity) we repeat analyses across three views of the data: first, the entire dataset; then, by political candidate followed amongst users who follow only one candidate; and finally, by the number of political candidates followed (degree of political engagement).

We hypothesize the following:

 FILL IT IN!!!!!! OR DO PER SECTION


\section{All Data}
\subsection{Hypotheses}
We hypothesize the following:

\begin{itemize} 
    \item \textbf{H1:} Story length has a \emph{negative} correlation with Twitter shares, due to the effects of the Internet attention economy and overexposure to political media \cite{goldhaber1997attention}
    \item \textbf{H2:} Emotionality has a \emph{positive} correlation with Twitter shares, consistent for viral content in general \cite{berger2012makes}
    \item \textbf{H3:} Positivity has a \emph{negative} correlation with Twitter shares, due to the nature of political news and contrary to generalized findings \cite{berger2012makes}

\end{itemize}

\subsection{Descriptives}
\subsection{Story Length}
\subsection{Emotionality}
\subsection{Positivity}

\section{By Candidate}
\subsection{Descriptives}
\subsection{Story Length}
\subsection{Emotionality}
\subsection{Positivity}


\section{By Degree of Political Engagement}
\subsection{Descriptives}
\subsection{Story Length}
\subsection{Emotionality}
\subsection{Positivity}

\section{Extra: Any relation with Trust?}
% see jun 10 update
% because it'd be funny not to check